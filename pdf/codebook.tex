%!TEX program = xelatex
\documentclass[a4paper,10pt,twocolumn,oneside]{article}
\setlength{\columnsep}{10pt}                                                                    %兩欄模式的間距
\setlength{\columnseprule}{0pt}                                                                %兩欄模式間格線粗細

\usepackage{amsthm}								%定義,例題
\usepackage{amssymb}
%\usepackage[margin=2cm]{geometry}
\usepackage{fontspec}								%設定字體
\usepackage{color}
\usepackage[x11names]{xcolor}
\usepackage{listings}								%顯示code用的
%\usepackage[Glenn]{fncychap}						%排版,頁面模板
\usepackage{fancyhdr}								%設定頁首頁尾
\usepackage{graphicx}								%Graphic
\usepackage{enumerate}
\usepackage{changepage}
\usepackage[compact]{titlesec}    %compact mode for reducing margin
\usepackage{amsmath}
\usepackage[CheckSingle, CJKmath]{xeCJK}
\usepackage{hyperref}
\hypersetup{
  linktoc=all,
  hidelinks
}
% \usepackage{CJKulem}

%\usepackage[T1]{fontenc}
\usepackage{courier}
\topmargin=-1pt
\headsep=5pt
\textheight=780pt
\footskip=0pt
\voffset=-40pt
\textwidth=545pt
\marginparsep=0pt
\marginparwidth=0pt
\marginparpush=0pt
\oddsidemargin=0pt
\evensidemargin=0pt
\hoffset=-42pt

%\renewcommand\listfigurename{圖目錄}
%\renewcommand\listtablename{表目錄} 

%%%%%%%%%%%%%%%%%%%%%%%%%%%%%

\setmainfont[Path=./fonts/Quicksand/]{Quicksand-Medium}
\setmonofont[Path=./fonts/RobotoMono/]{RobotoMono-Medium}
\setCJKmainfont[Path=./fonts/open-huninn-font/]{jf-openhuninn-1.1}           %中文字型
\XeTeXlinebreaklocale "zh"						%中文自動換行
\XeTeXlinebreakskip = 0pt plus 0pt				%設定段落之間的距離
\setcounter{secnumdepth}{3}						%目錄顯示第三層

%%%%%%%%%%%%%%%%%%%%%%%%%%%%%
\makeatletter
\lst@CCPutMacro\lst@ProcessOther {"2D}{\lst@ttfamily{-{}}{-{}}}
\@empty\z@\@empty
\makeatother
\lstset{											% Code顯示
language=C++,										% the language of the code
basicstyle=\footnotesize\ttfamily, 						% the size of the fonts that are used for the code
%numbers=left,										% where to put the line-numbers
numberstyle=\footnotesize,						% the size of the fonts that are used for the line-numbers
stepnumber=1,										% the step between two line-numbers. If it's 1, each line  will be numbered
numbersep=5pt,										% how far the line-numbers are from the code
backgroundcolor=\color{white},					% choose the background color. You must add \usepackage{color}
showspaces=false,									% show spaces adding particular underscores
showstringspaces=false,							% underline spaces within strings
showtabs=false,									% show tabs within strings adding particular underscores
frame=false,											% adds a frame around the code
tabsize=2,											% sets default tabsize to 2 spaces
captionpos=b,										% sets the caption-position to bottom
breaklines=true,									% sets automatic line breaking
breakatwhitespace=false,							% sets if automatic breaks should only happen at whitespace
escapeinside={\%*}{*)},							% if you want to add a comment within your code
morekeywords={*},									% if you want to add more keywords to the set
keywordstyle=\bfseries\color{Blue1},
commentstyle=\itshape\color{Red4},
stringstyle=\itshape\color{Green4},
literate={\ \ }{{\ }}1
}

%%%%%%%%%%%%%%%%%%%%%%%%%%%%%

\def\footnotesize{\fontsize{8}{9.5}\selectfont}
\titlespacing*{\section} {0pt}{2ex plus 1ex minus .2ex}{0.8ex plus .2ex}  % minimize margin
\titlespacing*{\subsection} {0pt}{1.75ex plus 1ex minus .2ex}{0ex plus .2ex} % minimize margin

\begin{document}
\pagestyle{fancy}
\fancyfoot{}
\fancyhead[L]{National Central University - __builtin_orz()}
\fancyhead[R]{\thepage}
\renewcommand{\headrulewidth}{0.4pt}
\renewcommand{\contentsname}{Contents} 

\scriptsize
\vspace{-2em}
\tableofcontents
\vspace{-1em}

%%%%%%%%%%%%%%%%%%%%%%%%%%%%%


\newcommand{\includecode}[2]{
  \subsection{#1}
  \vspace{-0.8em}
  \lstinputlisting{#2}
  \vspace{-1.2em}
}

\newcommand{\includetex}[2]{
  \subsection{#1}
  \input{#2}
  \vspace{-1.2em}
}

\newcommand{\sectiontitle}[1]{
  \section{#1}
  \vspace{-0.5em}
}


%%%%%%%%%%%%%%%%%%%%%%%%%%%%%

\sectiontitle{Basic}
\includecode{default}{../code/Basic/default.cpp}


\sectiontitle{Matching \& Flow}

\sectiontitle{Graph}
\includecode{2-SAT}{../code/Graph/TwoSAT.cpp}

\sectiontitle{Data Structure}
\includecode{Blackmagic}{../code/DataStructure/Blackmagic.cpp}
\includecode{Centroid Decomposition}{../code/DataStructure/CentroidDecomposition.cpp}

\sectiontitle{Math}
\includecode{CRT}{../code/Math/CRT.cpp}
\includecode{Factorize}{../code/Math/Factorize.cpp}
\includecode{NTT}{../code/Math/NTT.cpp}
\includecode{Lucas}{../code/Math/Lucas.cpp}
\includecode{FloorSum}{../code/Math/FloorSum.cpp}

\sectiontitle{Geometry}
\includecode{Convex Hull}{../code/Geometry/ConvexHull.cpp}
\includecode{Dynamic Convex Hull}{../code/Geometry/DynamicConvexHull.cpp}
\includecode{Half Plane Intersection}{../code/Geometry/HalfPlaneIntersection.cpp}
\includecode{Minimal Enclosing Circle}{../code/Geometry/MinimalEnclosingCircle.cpp}
\includecode{Minkowski}{../code/Geometry/Minkowski.cpp}

\sectiontitle{Stringology}
\includecode{Z-algorithm}{../code/String/Zalgorithm.cpp}
\includecode{Manacher}{../code/String/Manacher.cpp}
\includecode{SuffixArray}{../code/String/SuffixArray.cpp}
\includecode{PalindromicTree}{../code/String/PalindromicTree.cpp}

\sectiontitle{Misc}

\end{document}
